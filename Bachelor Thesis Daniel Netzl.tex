\documentclass{imc-inf}

\title{Implementation and analysis of a machine learning approach to long-term values investing}
\subtitle{Minimize risk while maximizing cash flow through stock picking based on fundamental company data}
\thesistype{Bachelor Thesis} % or Bachelor Expos\'e
\author{Daniel Netzl}
\supervisor{Dr. Supervisor Musterfrau}
\copyrightyear{2021}
\submissiondate{19.04.2023}
\keywords {Data Analytics, Machine Learning, Stock Market, Value Investing, Data Mining} % add keywords here 



% \usepackage{xyz}
% ... add your own packages here!
\usepackage{listings}
\usepackage{subcaption}
                              

\begin{document}
\frontmatter\maketitle{}


\begin{declarations}\end{declarations}



\begin{abstract}
	Abstract paragraphs should be unindented. Abstract text must fit on a single page. Try to present the essence of your work here. 
	
	According to Wikipedia\footnote{https://en.wikipedia.org/}, An abstract is a brief summary of a research article, thesis, review, conference proceeding, or any in-depth analysis of a particular subject and is often used to help the reader quickly ascertain the paper's purpose \cite{988366}. When used, an abstract always appears at the beginning of a manuscript or typescript, acting as the point-of-entry for any given academic paper or patent application. Abstracting and indexing services for various academic disciplines are aimed at compiling a body of literature for that particular subject.
	
	It is usually not a good practice to include references and footnotes in an abstract. Abstracts must be independent of other works, concise and complete in itself. 
	
	It is also possible to write structured abstracts. These are abstracts with distinct, labeled sections (e.g., Introduction, Methods, Results, Discussion), which makes it easier for the reader to navigate easily through the content. 
	
\end{abstract}


\addtoToC{Table of Contents}%
\tableofcontents%
\clearpage


\addtoToC{List of Tables}%
\listoftables
\clearpage


\addtoToC{List of Figures}%
\listoffigures
\clearpage


%   MAIN MATTER  %%%%%%%%%%%%%%%%%%%%%%%%%%%%%%%%%%%%%%%%%%%%%%%%%%%%%%%%%%%%%%
\mainmatter%

\chapter{Motivation}\label{chap:motivation}
When it comes to investing in the financial market, there are numerous approaches and tactics to consider. Investors strive for long-term success, but they frequently fail for a variety of reasons. Younger investors, in particular, have a tendency to underestimate the risk and end up accepting a significant financial loss. That has been more common in recent years, with a significant portion of generation Z following self-proclaimed “gurus” and their “expert” opinions and recommendations on numerous internet platforms. Following those incoherent investing schemes is tantamount to speculation or outright gambling. The promise of quick and cheap returns attracts investors, who fall prey to Wall Street's countless fads. In this paper, the author has two objectives. The first half of the paper describes the common mistakes made by investors and the challenges they confront, particularly in the present era of easy access to financial markets. The author provides his approach to asset management, as well as an algorithm that may aid him in executing it, in the hopes of assisting investors in recognizing and so avoiding these losing methods ~\cite{margin_of_safety}.

The author will advocate one specific investing technique for the remainder of the paper: the value-investment philosophy. This idea encapsulates the technique of investing in assets that trade at a significant discount to their intrinsic value. The strategy has been used for a long time, with investors experiencing minimal risk and good rewards ~\cite{margin_of_safety}.
To achieve investment success, it is of utmost importance to know where others go wrong and deliberately choose a path to avoid those pitfalls. The thesis will mainly be built upon the most honorable representatives of value investing, including Benjamin Graham, David Dodd, and Seth Klarman. 

Security Analysis ~\cite{security_analysis}, written by Benjamin Graham and David Dodd more than fifty years ago, is widely considered as the bible of value investing. For generations of value investors, that single work has paved the road. Graham's most recent book, The Intelligent Investor ~\cite{the_intelligent_investor}, is a less scholarly account of the value-investing process. Warren Buffett, the chairman of Berkshire Hathaway, Inc., and a Graham student, is widely recognized as the most successful value investor today ~\cite{margin_of_safety} Seth Klarman published the most recent book this thesis’ methods are based on. With Margin of Safety ~\cite{margin_of_safety} Klarman emphasizes the necessity of avoiding typical blunders. By describing his approach to value investing, he demonstrates that success in the financial markets requires a defined strategy backed by patience, ambition, and hard effort.

\chapter{Problem definition}\label{chap:problem defintion}
It's terrifying to see how many naïve and ingenious investors have had horrible financial outcomes. If this paper and its algorithm succeed in their approach, the author will be overjoyed if he can persuade even a few of the readers to avoid risky investment selections in favor of sensible ones that will safeguard and keep their hard-earned cash.
Investors are frequently their own worst adversaries. On the one hand, when price trends are rising, investors are more likely to speculate and follow their emotional greed, placing high-risk bets based on optimistic expectations and ignoring related danger. When prices are declining, on the other hand, emotions again play a huge role. Fear of loss causes the investors to concentrate solely on the prices continuing to fall, rather than on the underlying data of the companies. Regardless of the current market scenario, many people are looking for a winning recipe. Reality, however, does not follow any mathematical equations.

Younger investors, in particular, are more likely to acquire their financial advice from dubious sources, such as influencers who claim to have had amazing success on Wall Street and know exactly what they are doing. Due to the ease of access to financial markets and the availability of super-cheap transactions provided by online brokers, a significant portion of Generation Z is perceived to be significantly involved in extremely speculative high-frequency trades. 
This effect has been particularly noticeable in recent years, when market prices have only shown one direction. The S\&P 500, for example, climbed by over 98 percent between May 2017 and January 2022. That's nearly a 20 percent annual increase. The NASDAQ 100 hit its interim peak around the same time, gaining roughly 190 percent, or 38 percent annually, in the same time frame. It goes without saying that many new investors were enticed by the supposedly easy and extraordinary gains.
However, as this paper is written, those new investors are experiencing their first baisse, revealing their expertise to be nothing more than riding a wave together with the rest of the market. It is critical to understand what one is doing and to have a clear approach during such times.

The strength of such speculative investors should not be underestimated. As can be observed in the case of the Gamestop stock, a downward-pointing company's stock price has risen by over a thousand percent in half a year, only to plummet by half immediately after (but still remain at a high level). There have been multiple instances where private investors have banded together on social media, particularly on the website reddit.com, to artificially inflate prices to unheard-of highs, enticing a large number of naive investors and leaving the vast majority of them with irreversible losses. Many individual and institutional investors overlook or deliberately disregard core corporate principles, perceiving stocks as nothing more than pieces of paper to be traded back and forth.

Investors must ultimately choose their preferred methods. Either they take a seemingly simple way that provides the comfort of consensus, or they take a path that involves emotional responses fueled by greed and fear and guided by short-term thinking ~\cite{margin_of_safety}.

Most people are unwilling to make the commitment required by the alternative. Those methods, which include value investing, involve fundamental analysis, which treats equities as fractional ownership of the underlying company they represent ~\cite{margin_of_safety}.

It is critical to distinguish between speculation and investing. Anyone who buys and sells stocks nowadays is referred to be an investor. Nonetheless, the vast majority makes no attempt to justify their investing decision. Most of the time, no evaluations are performed, and stocks are bought and sold when markets rise and fall. The recent trend of the stock price is frequently used as a buying criterion. If the stock outperformed the market, it gets purchased. If any analysis is conducted, they frequently include a review of long-term past growth that is expected to continue. Also, "investors" may select companies that have not yet produced spectacular outcomes but are expected to do so in the future. Growth stocks and assets from the technological or health-care sectors are common in these companies. "Investors" hope to benefit from enormous future results ~\cite{the_intelligent_investor}. 

The "investor" faces two distinct dangers in his search for the most promising stocks. He or she could be wrong about the company's future progress. Even if he is correct, the present market price may already reflect the anticipated development. Insofar as they are predictable, a company's near-future results are often already taken into account. By making a judgment based on those criteria, one is likely to discover that others have already done so. To summarize, in order to obtain above-average results, one must adhere to policies that are essentially sound and promising, even if they are unpopular on Wall Street ~\cite{the_intelligent_investor}.
Value investing aims to identify stocks that have been overlooked and are consequently undervalued. However, it is not so straightforward, since the process requires a lot of patience. Selling an overrated and overly popular issue takes boldness and endurance. The theory is sound, and while successful application is not impossible, mastering it is a difficult art ~\cite{the_intelligent_investor}. Even more so nowadays when stock prices are adopted in a fraction of a second.

Even yet, the concept of value investing is unlikely to turn anyone into a profitable value investor. Hard work and tight discipline are required for value investment. Only a small percentage of people are willing and able to devote the necessary time and effort, and only a small percentage of people have the right mindset to be successful in the long run. Because those virtues are becoming increasingly rare as the modern environment becomes more dynamic, the algorithm under investigation tries to aid in making value investing accessible to a wider audience.

Naturally, this paper will not present a foolproof investment method. It will not guarantee any profits in advance, but it will highlight the personal risk that everyone must analyze before making investing decisions. The presented theory, as well as the algorithm developed, do not offer any financial advice or suggestions. The algorithm's signals are nothing more than the results of various calculations that, according to the inventor, might be utilized to aid in the discovery of undervalued companies. It is entirely up to the readers to decide how they will use the material.


\section{State of the art}
Most relevant state of the art/state of practice
Mention what is done by Firstauthorlastname et al. and what is needed to be done
If needed you can refer to multiple related works

\section{Background}
Background knowledge needed to understand your model
Briefly describe the methods that will be used in your model

\section{Model}

Describe how your model or approach will work
Add a diagram about the model so that it helps audience to understand how it will work

\section{Experimental Setup/Implementation}
This slide can be one of the two types: experimental setup for data science or implementation details for tool development
Experimental Setup:
Plan -> how you will setup your experiment
Optional -> if needed describe how you will define threshold
Implementation details:
Plan -> how you will develop the tool

\section{Evaluation Plan}
Plan -> how you will evaluate the developed tool or the model

\section{Conclusion/Summary}

Mention what will be the potential contribution of your thesis
Repeat how the research questions will ba answered and/or how your research goals will be achieved


\chapter{Example Chapter}
This is only an example of a chapter! Anyways, all thesis should have a problem statement -- not necessarily as a separate chapter though. Only after you know the problem, it will be possible for you to evaluate the results of what you did. If you want to see examples of evaluations, have a look at how graph visualizations are evaluated here \cite{DBLP:journals/access/BurchHWPWH21}. 

\section{Code and syntax highlighting}

You may sometimes want to add code snippets to your thesis. You can do so by using \texttt{lstlisting}. Use this with care, as code should not be extensively presented in the thesis. Here is an example. 

\begin{lstlisting}[language=Python]
def addition ():
    print("I am adding numbers here!")
    n = float(input("Enter the number: "))
    t = 0 // Total number enter
    ans = 0
    while n != 0:
        ans = ans + n
        t+=1
        n = float(input("Enter another number (0 to end): "))
    return [ans,t]
\end{lstlisting}

\section{Labels and References}
See \autoref{chap:introduction} for interesting stuff and see a cool logo in \autoref{fig:logo}. If you are still not convinced, try adding a footnote\footnote{did you like it?}. Its easy to add citations, just use a bibtex file to list your references and cite them here like this~\cite{988366}. If you want to read a cool paper~\cite{DBLP:conf/euromicro/DhunganaHW20}, just contact the author of the paper. Haha, that was funny! 


\section{Mathematical Equations and Expressions}
Basic equations in  \LaTeX{} can be easily "programmed". Fermat's Last Theorem (sometimes called Fermat's conjecture, especially in older texts) states that no three positive integers a, b, and c satisfy the equation \[ a^n + b^n = c^n \] for any integer value of $n$ greater than $2$. The cases $n = 1$ and  $n = 1$  have been known since antiquity to have infinitely many solutions. And because its so much fun, here is an integral for you - thank me later!  

\[ \int\limits_0^1 x^2 + y^2 \ dx \]

Do you want a more complex formula, I have no idea what it means, but it looks pretty. 

\[\oint_{i=1}^n \sum_{i=1}^{\infty} \frac{1}{n^s} 
= \prod_p \frac{1}{1 - p^{-s}} \]


\section{Enumerations and Descriptions}
Here is a simple list: 
\begin{enumerate}
	\item The labels consists of sequential numbers.
	\item The numbers starts at 1 with every call to the enumerate environment.
\end{enumerate}

Here is another list: 

\begin{enumerate}
	\item The labels consists of sequential numbers.
	\begin{itemize}
		\item The individual entries are indicated with a black dot, a so-called bullet.
		\item The text in the entries may be of any length.
	\end{itemize}
	\item The numbers starts at 1 with every call to the enumerate environment.
\end{enumerate}

Maybe such descriptions are also useful. These look neat to me. What do you think? Oh, I forgot, this document is not a tutorial. 
\begin{description}
	\item[Short] This is a shorter item label, and some text that talks
	about it. The text is wrapped into a paragraph, with successive
	lines indented.
	\item[Rather longer label] This is a longer item label.  As you can
	see, the text is not started a specified distance in -- unlike
	with other lists -- but is spaced a fixed distance from the end
	of the label.
\end{description}



\section{Adding images}
Adding a simple image is easy. Adding complex images is also easy. What is a complex image anyway? 
\begin{figure}[h]
	\centering
	\includegraphics[width=1.0\textwidth]{imclogo.png}
	\caption{IMC Logo}
	\label{fig:logo}
\end{figure}





\begin{figure}[ht]
	\begin{subfigure}{.5\textwidth}
		\centering
		% include first image
		\includegraphics[width=.8\linewidth]{imclogo.png}  
		\caption{Put your sub-caption here}
		\label{fig:sub-first}
	\end{subfigure}
	\begin{subfigure}{.5\textwidth}
		\centering
		% include second image
		\includegraphics[width=.8\linewidth]{imclogo.png}  
		\caption{Put your sub-caption here}
		\label{fig:sub-second}
	\end{subfigure}
	\caption{Including sub images! }
	\label{fig:fig}
\end{figure}

\section{Tables}

\begin{table}[ht]
\begin{tabular}{ |p{3cm}||p{3cm}|p{3cm}|p{3cm}|  }
	\hline
	\multicolumn{4}{|c|}{Country List} \\
	\hline
	Country Name     or Area Name& ISO ALPHA 2 Code &ISO ALPHA 3 Code&ISO numeric Code\\
	\hline
	Afghanistan   & AF    &AFG&   004\\
	Aland Islands&   AX  & ALA   &248\\
	Albania &AL & ALB&  008\\
	Algeria    &DZ & DZA&  012\\
	American Samoa&   AS  & ASM&016\\
	Andorra& AD  & AND   &020\\
	Angola& AO  & AGO&024\\
	\hline
\end{tabular}
\caption{\label{tab:table-name}Example table}
\end{table}




%   BACK MATTER  %%%%%%%%%%%%%%%%%%%%%%%%%%%%%%%%%%%%%%%%%%%%%%%%%%%%%%%%%%%%%%
%
%   References and appendices. Appendices come after the bibliography and
%   should be in the order that they are referred to in the text.
%
%   If you include figures, etc. in an appendix, be sure to use
%
%       \caption[]{...}
%
%   to make sure they are not listed in the List of Figures.
%

\backmatter%
	\addtoToC{Bibliography}
	\bibliographystyle{IEEEtran}
	\bibliography{references}
	

\begin{appendices} % optional
\chapter{Example Appendix 1}

Appendices should be used for supplemental information that does not form part of the main research. Remember that figures and tables in appendices should not be listed in the List of Figures or List of Tables. 

\chapter{Example Appendix 2}

Appendices should be used for supplemental information that does not form part of the main research. Remember that figures and tables in appendices should not be listed in the List of Figures or List of Tables. 
	
\end{appendices}
\end{document}
